\usepackage{indentfirst}
%	adds indent to first paragraph (something Latex does not normally do)

\usepackage{csquotes}
%	Context Sensitive Quotes - will convert straight quotes to the appropriate smart quotes automatically.

\usepackage{titletoc,tocloft}

\setcounter{secnumdepth}{0}

\setlength{\parindent}{2em}
%	adds an indent to the paragraphs

\linespread{1.25}
%	line spacing

\usepackage{titlesec}
\titlespacing{\subsubsection}{0pt}{*1}{*-4}

\usepackage{ragged2e}

\usepackage{geometry}
\geometry{
	margin=1.0in,
%	top=0.75in,
	headheight=1.80cm,
	headsep=0.75ex,
	%footsep=0.5in,
	footskip=0.5in,
	bottom=1.0in
}
\savegeometry{normal}

\usepackage{graphicx}
\usepackage[space]{grffile}
%	this allows spaces to be used the file names and paths of images

\DeclareGraphicsExtensions{.png,.pdf}

\usepackage{float}
\renewcommand*\listfigurename{Images}
%cof changing list of figures title, but not necessary when creating own float type
\floatstyle{plaintop}
\newfloat{imageflt}{H}{loi}

\usepackage{caption}
\captionsetup{
	skip=-0.5\baselineskip,
	labelformat=empty,
}
%this will set the skip between the caption and figure to be effectively missing for all floats

\DeclareCaptionStyle{imageflt}{
	listformat=empty,
	justification=centering,
}
\captionsetup[imageflt]{
	style=imageflt,
}

\newcommand{\image}[2][width=1.0\textwidth, totalheight=0.33\textheight]{
%	the first argument is optional with the default value given above in the second set of square brackets, so no if/then statement is needed
	\def\options{#1}
	\def\file{#2}
	
	\begin{imageflt}[H]
		\caption*{\textbf{#2}}
		%\captionlistentry{#2}
%
		\vspace{2ex}	\hrule	\nobreak	\vspace{2pt}
	%	puts space above and below the rule, while also preventing there from being a page break between the rule and the image
		\hfill	%\hspace{-2em}
	%	subtracting 2em as this is the indent length that I cannot otherwise remove
			\includegraphics[keepaspectratio,#1]{{#2}}	\nobreak
		\hspace*{\fill}
	%	for some reason this works better than hfill
		\vspace{2pt}	\hrule	\vspace{2ex}
	\end{imageflt}
}
%by using Figures I can have the titles easily added and add \listoffigures to show all of the figures in a section

\usepackage{hyperref}
%	the package for hyperlinks and setting their color, highlight, etc.

\hypersetup{
	colorlinks,	%removes the box around links
	linkcolor=black,
	filecolor=black,
	urlcolor=blue,
%	linktoc=all,	%makes the name and page numbers links
}
\newcommand{\changeurlcolor}[1]{\hypersetup{urlcolor=#1}}

\usepackage{qrcode}


\usepackage{listings}
%	this is the package that will properly read and apply a special style to code

\definecolor{codeBackground}{RGB}{242,242,242}
\definecolor{comments}{RGB}{0,100,0}
\definecolor{keywords}{RGB}{0,0,200}

\lstset{
	breaklines			=	true,
	breakatwhitespace	=	true,
	breakautoindent		=	true,
	breakindent			=	1em,
	tabsize				=	4,
	lineskip			=	-0.5ex,
	frame				=	tb,
	backgroundcolor		=	\color{codeBackground},
	commentstyle		=	\color{comments},
	keywordstyle		=	\color{keywords}\nolinebreak,
	sensitive			=	true,
	aboveskip			=	3ex,
	belowskip			=	2ex,
	showstringspaces	=	false,
}
%	these options will be set for all languages used

\lstnewenvironment{styleR}
{
	\noindent
	% \minipage{\textwidth}
	\lstset{
		language			=	R,
		morekeywords		=	{PERIODS, diff.CONS, labelBreak, sci2norm, rem_, round2, nearCEIL, nearFLOOR, stats, sepCOL, remUNI, unitCOL, tempCROSS, CPUslopes, slope, sinkTXT, printFrame, writeOCC, OCCHTML, sinkHTML, customSave, TEMP_point, SOCK_point, CORE_point, unCORE_point, FREQ_point, themeSCALES, graphMEAN, graphMAX, graphFREQ, graphPOWER, graphHIST, FREQspec_line},
	}
	%	by specifying the language here, it will load up its options, but still use the options set above
}	{
	% \endminipage
}

\lstnewenvironment{stylePy}
{
	\noindent
	% \minipage{\textwidth}
	\lstset{
		language			=	Python,
		morekeywords		=	{TESTfun, lnkCheck, INPUT, CINEBENCHr20, CINEBENCHr23, _3DMARK, Prime95, PULSE, PULSEx, PULSEk, kill, timeFUT, codFind, numFind, codGraph, numGen},
	}
	%	by specifying the language here, it will load up its options, but still use the options set above
}	{
	% \endminipage
}

\newcommand{\listingfile}[3]{
	% \lstset{language=#1, caption={[\namesection: Code \thelstlisting]}}
	% \lstset{language=#1, caption={Code \thelstlisting}}
	\clearpage
	{\lstset{language=#1}
	
	\noindent
	% \begin{minipage}{\textwidth}
		\subsection{#3}
		% \textbf{#3}
	
		\lstinputlisting{"#2/#3"}
	% \end{minipage}
	}
}

\newcommand{\listingfrag}[5]{
	% \lstset{language=#1, caption={[\namesection: Code \thelstlisting]}}
	% \lstset{language=#1, caption={Code \thelstlisting}}
	\clearpage
	{\lstset{language=#1}
	
	\noindent
	% \begin{minipage}{\textwidth}
		\subsection{#3 (Fragment)}
		% \textbf{#3}
	
		\lstinputlisting[firstline = #4, lastline = #5]{"#2/#3"}
	% \end{minipage}
	}
}

% \renewcommand{\contentsname}{\hfill \bfseries Table of Contents \hfill}
\renewcommand{\contentsname}{\bfseries{Table of Contents}}

\setlength\cftbeforetoctitleskip{0ex}
\setlength\cftaftertoctitleskip{0ex}
%	removes the vertical spacing above and below the Table of Contents header

\renewcommand{\cftsecfont}{}					%	clears any special font styles
\setlength{\cftbeforepartskip}{3.0ex}			%	vertical spacing between Parts
\setlength{\cftbeforesecskip}{0.5ex}			%	vertical spacing between Sections
\renewcommand{\cftsecleader}{\cftdotfill{4}}	%	adds dots and sets the dot separation
\renewcommand{\cftsecpagefont}{}				%	clears any special font styles for the page number

\renewcommand{\cftsubsecfont}{\itshape}			%	makes the subsection entries italicized
\setlength{\cftbeforesubsecskip}{0.75ex}		%	vertical spacing
\renewcommand{\cftsubsecleader}{\cftdotfill{8}}	%	adds dots and sets the dot separation
\renewcommand{\cftsecpagefont}{}				%	clears any special font styles for the page number
\cftsetindents{subsection}{2em}{2em}			%	sets the indent for the entry and the numwidth

\cftpagenumbersoff{part}	%	removes page numbers from the Part

\newcommand{\Part}[1]{
	\phantomsection
	\addcontentsline{toc}{part}{#1}
%	\renewcommand{\namepart}{#1}
%		for fancy headings I am not using (see GLSC Curriculum)
}
%	this will set the name of the part and add it to the TOC without adding the massive header to the page

\newcommand{\DBS}{\textbackslash\textbackslash}
\newcommand{\SBS}{\textbackslash}
\newcommand{\n}{\textbackslash{n}}
\renewcommand{\r}{\textbackslash{r}}
\renewcommand{\>}{\textgreater}
\newcommand{\<}{\textless}
\newcommand{\til}{$\sim$}


\widowpenalty10000
\clubpenalty10000

\newcommand{\subCustomFunction}{\addcontentsline{toc}{subsubsection}{Custom Functions}}